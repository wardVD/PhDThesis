%----------------------------------------------------------------------------------------
%	Dankje
%----------------------------------------------------------------------------------------
\pagenumbering{gobble}% Remove page numbers (and reset to 1)
\clearpage
\thispagestyle{empty}
%\pagestyle{empty} % No headers

\begin{huge}
\textbf{Dankwoord - Acknowledgements}\\
\end{huge}
\begin{flushright}
\textit{All I can be is me - whoever that is $\sim$ Bob Dylan
\\}
\end{flushright}

- Blub

\iffalse
\noindent Ondanks het maandenlang schrijven aan dit werk valt het me op hoe lastig het is om dit stukje op papier te zetten. Ik denk dat de meesten mij wel kennen als iemand die er de humor probeert in te houden, wat het niet evidenter maakt om een aantal mensen oprecht te bedanken. Ik kan tevens ook niet garanderen dat onderstaande tekst niet tot gefronste wenkbrauwen zal leiden. Ik blijf hoe dan ook de auteur van dit dankwoord.\\

\noindent Allereerst wilt ik natuurlijk mijn promotor professor Ryckbosch bedanken. Mijn eerste cursus deeltjesfysica is ondertussen al bijna acht jaar geleden en ik herinner mij hem nog maar vaag. Dat professor Ryckbosch mijn blik aanzienlijk verruimd heeft behoeft geen betoog. Ondanks het feit dat we niet meer in dezelfde gang zullen werken mogen, wat mij betreft, de niet-altijd-zo-serieuze-e-mails die zich sporadisch in mijn inbox bevonden zich blijven opstapelen. Dank u voor de voorbije jaren!\\

\noindent Thanks to the people from the BSM group for listening and helping in this work. Many thanks to Anna for all the times I had questions and for the abundunce of useful tips in how I could progress in my work. Thank you Frederik and Sarah for the nice times we had at the meetings. I'll come visit Wuppertal again at some point. I promise.\\

\noindent Merci, Antoine, pour ces dernières années durant lesquelles tu as voulu partager un bureau avec moi. Nous en avons vu beaucoup aller et venir, mais heureusement nos parcours se sont chevauchés suffisamment longtemps afin de me permettre un profond apprentissage du français. Bien sûr, ma copine a dû traduire ce petit mot parce que tu n'as jamais vraiment pris ton travail de professeur de français au sérieux. Je te souhaite encore beaucoup de succès dans ta carrière !\\

\noindent Aan alle anderen die ik de voorbije jaren collega's heb mogen noemen: bedankt voor de familiale sfeer die we op het INW mochten ervaren, de constante in- en uitstroom van mensen maakte dat er niet eenvoudiger op. Er zijn er zoveel waar ik de laatste jaren veel aan gehad heb, maar C\'eline, Ianthe en Giel: super bedankt voor de vele leuke gesprekken. Merci voor het nalezen, Stef, en nog veel succes de komende jaren.\\

\noindent Uiteraard hoor ik mijn familie en vooral mijn ouders te bedanken. Ik praat niet vaak over mijn werk in jullie bijzijn - vrije tijd is vrije tijd - dus ik wil jullie vooral bedanken om te luisteren naar al die andere onzin die uit mijn mond kwam tijdens onze etentjes en familiefeesten en om begrip te tonen toen ik voor de zoveelste keer zei dat we niet over mijn werk hoefden te babbelen. Jullie hebben mij misschien een beetje te vaak ``het is zo abstract dat het geen nut heeft om het te proberen uitleggen'' horen zeggen. Ik zal beter mijn best doen. Oh, en sorry voor mijn gedrag tijdens de examens. Die tijden zijn gelukkig voorbij.\\

\noindent De laatste jaren heb ik zoveel vrienden gemaakt dat ik bijna niet weet waar ik moet beginnen. Misschien hou ik mij het best bij diegenen waarvan iedereen weet dat ik ze niet uit mijn thesis kan houden. Dieter, Kris en Sam, er zijn nu al meer dan 16 jaar verstreken sinds het eerste middelbaar... Jullie weten ook dat het onmogelijk is om de goede dingen allemaal te beginnen opsommen. Ondanks het feit dat we elkaar al zo lang kennen, kijk ik uit naar de vele keren dat we nog zullen afspreken, hoe we nostalgisch gaan terugdenken en hoe we elkaars gedrag nog verder in het belachelijke gaan trekken.\\

\noindent James, mijn ventje, dit werk zou nooit volledig af zijn zonder de meest goedhartige persoon van het Waasland te vernoemen. Een beetje melig misschien, maar ik hoor je al snikkend lachen wanneer je voor het eerst zal lezen dat ik dit effectief heb neergepend. Dat je me nog veel goede raad mag geven en hopelijk vinden we beiden nog de tijd en goesting om te blijven fietsen.\\

\noindent Hannes, ik heb je fysiek zien opgroeien als 15-jarig snotneusje die toen al groter was dan ikzelf (ik zie je nog steeds in de nijlpaardkamer liggen toen je al 1m90 was en je je heup gebroken had). Mentaal mag je nog wat rijpen, maar ik ben alleszins al blij dat je mijn oor niet meer probeert te likken als we gaan fietsen. Er valt nog veel meer te zeggen, maar dat gaan we de lezers misschien best niet aandoen. Je bent er altijd voor me geweest en dat apprecieer ik nog meer dan ik soms doe uitschijnen.\\

\noindent Mijn klein Angeke, ik weet dat ik zei dat ik het leuk vond dat je veel babbelt toen we elkaar nog maar net kenden, maar zo had ik het me nu ook weer niet voorgesteld. Nee, schat, wat ben ik blij dat je het nog steeds met me uithoudt. Je bent er een groot deel van mijn doctoraat geweest en hebt me de nodige keren de duw in de rug kunnen geven. Mijn beste vriendin en mijn grootste liefde, team Wangeke doet dat goed en laat dit maar \'e\'en van de eerste stappen in ons hele lange verhaal worden.
\fi

~\vfill

\newpage
