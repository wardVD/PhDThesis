\chapter{Motivation of the Analysis}
\label{ch:theoreticalmotivation}
\begin{flushright}
\textit{\\The world is divided into people who think they are right $\sim$ Anonymous\\}
\end{flushright}

\noindent As seen in Chapter \ref{ch:SM}, there is an ongoing debate which beyond-the-Standard-Model physics models could help explain questions we do not have answers for. Over the last decades, this quest has proven to be non-trivial since many accelerator experiments have not given any clear hints towards physics that cannot be explained by the Standard Model. A big part of the physics community is trying its best to help answer these riddles and dedicated experiments have been constructed in the search for new physics. Other collaborations try to make use of their detector in the most efficient way possible. These experiments most often try to look for BSM physics by searching for signals in their detector that could not be explained by the particles we know today. One example, and also the subject of this work, is to try to look for particles that have a lower electromagnetic, but non-zero, charge than the charged particles of the Standard Model.

\section{Introduction}
As seen in Chapter \ref{ch:SM}, all free particles have an electromagnetic charge that is a multiple of the absolute electron charge, $e$, equal to $1.602 \times 10^{-19}$ C. Elementary particles such as (anti)quarks have fractional charges equal to $\pm\frac{1}{3}e$ and $\pm\frac{2}{3}e$, but have never been seen as isolated particles due to \textit{confinement} as explained in Section \ref{sub:quarks}. No other particles are expected to have a charge lower than $e$ and are therefore perfect candidates for searches beyond the Standard Model. Different experiments have sought these anomalously charged particles, which are referred to as \textit{Lightly Ionizing Particles (LIPs)} or \textit{Stable Massive Particles (SMPs)}. Throughout this work the latter denomination is used, indicating they do not rapidly decay and have masses significantly higher than the lightest leptons.

\section{Theory}
In Section \ref{sub:unifying}, possible extensions of the Standard Model were already introduced. One of the simplest possible extensions of the SU(3) $\times$ SU(2) $\times$ U(1) groups is the SU(5) gauge group. It is the smallest Lie group that can contain the group of the Standard Model without introducing any new fermions. It could explain charge quantization \cite{He:1989eq}, has complex representations and can accommodate fractional charges. In this scheme, new vector bosons, usually called $X$ and $Y$ bosons, occur with charges $\frac{4}{3}$ and $\frac{1}{3}$. Extensions of the SU(5) models allow for color singlet particles with charges $\frac{1}{3}$ and $\frac{2}{3}$ \cite{Barr:1982vj}. Other possible extensions are based on the SU(7) \cite{Frampton:1982gc}, SU(8) \cite{Yu:1984pb}, SO(14) \cite{Yamamoto:1982sk}, SO(18) \cite{Dong:1983nh}, SO(10) $\times$ SO(8) groups \cite{Jiang:1985jy}. 

It should be noted that the simplest form of an SU(5) gauge group is already highly constrained as proton decay is allowed in this model and estimated to be around $10^{30}-10^{31}$ years. However, experimental results have shown the lifetime to be $>1.6 \times 10^{34}$ years $\left(\tau\left(p \rightarrow \pi^0 e^+\right)\right)$ and $>7.7 \times 10^{33}$ years $\left(\tau\left(p \rightarrow \pi^0 \mu^+\right)\right)$ \cite{Miura:2016krn}.\\
\newline
There are also some string theories where massive particles with a fractional charge are predicted \cite{Wen:1985qj,Antoniadis:1992eb}. This was later confirmed to occur very often in certain compactifications \cite{Athanasiu:1988uj}.\\

\noindent More recently, there has been an increasing interest in searches for millicharged particles. New particles could couple to the Standard Model via a ``kinetic mixing'' or ``hypercharge portal'' \cite{Holdom:1985ag,Izaguirre:2015eya}. And in recent years, they were studied as possible candidates for dark matter \cite{Brahm:1989jh,Boehm:2003hm,Pospelov:2007mp,Bjorken:2009mm}. However, the charges of these particles are often $<10^{-3}e$ and therefore no ideal candidates in neutrino Cherenkov experiments. It is possible to look for them in neutrino experiments \cite{Magill:2018tbb}, but more targeted toward future experiments such as DUNE \cite{Acciarri:2015uup} and SHiP \cite{Anelli:2015pba}. A more detailed explanation of these particles can be found in Ref. \cite{Battaglieri:2017aum}. The most stringent upper limit in millicharged particles known to the author is given in Ref. \cite{Alvis:2018yte}.\\

\noindent There are many other possible extensions, but these go beyond the scope of this work. One should just keep in mind that no free particles with an anomalous charge less than $e$ are expected in the SM and that, if seen, they give clear hints of beyond-the-Standard-Model physics and would help in finding a more clear picture of what is possibly hidden beyond the realms of our understanding.


%\url{https://ac.els-cdn.com/S0370269314001257/1-s2.0-S0370269314001257-main.pdf?_tid=7f8b3af4-8845-417fa219-703f382cb092&acdnat=1535556263_162a6333d0a615412be21aa5fc3e5720
%}



\section{Previous searches}
\label{sec:prevsearches}
There are several ways one can assume to produce fractional charge particles. Different assumptions lead to different possible searches with previous and current detectors. In the following, the results of several experiments are shown. Accelerator and fixed target experiments look for particles that might be created in particle collisions, resulting in upper limits for production cross sections if no candidate events are found. Telescope experiments, on the other hand, often assume a flux of incoming particles that is bound to an upper limit if no candidate events are found.
%More information and a very good overview can be found in ..... diee summarypaper????

\subsection{Searches with accelerators and fixed targets}
The total energy of the interaction should be large enough to produce particles of a certain mass. The square of the center of mass energy is given by:

\begin{equation}
\label{eq:totalenergy}
\begin{split}
s &= \left(p_1 + p_2\right)^2\\
&= m_1^2 + m_2^2 + 2E_1E_2 - 2\vec{p_1}\vec{p_2},\\
\end{split}
\end{equation}
where $p_{1,2}$ are the four-momenta of the two particles and $c$, the speed of light, is set to 1 in natural units.
Assuming $E$ is the energy of the incoming particle with mass $m_i$ and $m_t$ the mass of a target particle in rest and $E >> m_t, m_i$, the maximal mass reach of a search is given by:

\begin{equation}
\label{eq:mmax}
m_{max} \approx \sqrt{2 m_t E}.
\end{equation}
If $I$ is the incoming particle from the input beam and $N$ a nucleus, the production of exotic particles can then be denoted as

\begin{equation}
I + N \rightarrow F + X,
\end{equation}
where $F$ stands for the fractional charged particle and $X$ for the other particles that are produced in the interaction. No experiments that used accelerators and fixed targets found evidence for the existence of fractional charge particles \cite{Lyons:1984pw}. The highest-energy search used muons with a beam energy of 200 GeV, resulting in an $m_{max}$ of 19 GeV \cite{Aubert:1983jy}.
\subsection{Colliders}
Particle colliders can reach much higher energies than most fixed-target experiments. We see from Eq. \ref{eq:totalenergy} that the maximal mass of new particles in a storage ring that is colliding particles of energy $E$ with $E>>m_1,m_2$ scales with the energy: 

\begin{equation}
\begin{split}
&s = 4E^2,\\
\rightarrow \ &m_{max} = 2E
\end{split}
\end{equation}
There is a big difference in lepton and hadron accelerator experiments since much less particles are being produced in the former due to the absence of strong interactions. The production is ``cleaner'' and the particles sought are easier to distinguish from other productions. But, it is more difficult to reach higher energies for lepton accelerators\footnote{The radiative power of synchrotron radiation scales with a factor of $m^{-4}$: particles with low mass lose much more energy in circular accelerators with a fixed radius.}. An overview of electron-positron colliders is given in Table \ref{tab:elecposcollider}. No evidence for fractionally charged particles was found.


\begin{table}[]
\caption{Highest-energy fractional charge particle searches in electron-positron colliders. No evidence for fractionally charged particles was found.}
\label{tab:elecposcollider}
\centering
\begin{tabular}{|l|c|c|c|}
\hline
\rowcolor[HTML]{F1A91E} 
\textbf{$\sqrt{s}$ (GeV)} & \textbf{Charges sought}                        & \textbf{Collider} &\textbf{Reference}\\ \hline
1-1.4		 & $\frac{2}{3}$						 & VEPP-2M  & \cite{Bondar:1985sb} \\ \hline
29			 & $\frac{1}{3},\frac{2}{3}$			 & PEP & \cite{Aihara:1984px} \\ \hline		
130-209      & $\frac{2}{3},\frac{4}{3},\frac{5}{3}$ & LEP & \cite{Abbiendi:2003yd}    \\ \hline
130-136, 161 and 172 & $\frac{2}{3}$                 & LEP & \cite{Abreu:1996py}    \\ \hline
91.2 ($m_Z$) & $\frac{2}{3},\frac{4}{3}$             & LEP & \cite{Akers:1995az}    \\ \hline
91.2 ($m_Z$) & $\frac{4}{3}$   	    			     & LEP & \cite{Buskulic:1992mr}  \\ \hline
\end{tabular}
\end{table}

Experiments that use proton-antiproton colliders have reached larger masses but have also found no evidence of fractionally charged particles. An overview is given in Table \ref{tab:protonantiprotoncollider}.


\begin{table}[]
\caption{Highest-energy fractional charge particle searches in proton-(anti)proton colliders. No evidence for fractionally charged particles was found.}
\label{tab:protonantiprotoncollider}
\centering
\begin{tabular}{|l|c|c|c|c|}
\hline
\rowcolor[HTML]{F1A91E} 
\textbf{$\sqrt{s}$ (GeV)} & \textbf{Type} & \textbf{Charges sought}            & \textbf{Collider} & \textbf{Reference} \\ \hline
540		 & $p\overline{p}$ & $\frac{1}{3},\frac{2}{3}$ & SPS      & \cite{Banner:1985ev} \\ \hline
1800          & $p\overline{p}$ & $\frac{2}{3},\frac{4}{3}$ & Tevatron & \cite{Abe:1992vr}         \\ \hline
1800          & $p\overline{p}$ & $\frac{1}{3},\frac{2}{3}$ & Tevatron & \cite{Acosta:2002ju} \\ \hline
7000          & $pp$ & $\frac{1}{3},\frac{2}{3}$ & LHC & \cite{CMS:2012xi} \\ \hline

\end{tabular}
\end{table}

A more recent search was performed at the LHC, a proton-proton collider, when operating at a center of mass energy of 7 TeV. No evidence of particles with fractional charge was found. An upper limit of 95\% confidence level was set for particles with electric charge $\frac{2}{3}$ up to a mass of 310 GeV and 140 GeV for those with charge $\frac{1}{3}$ \cite{CMS:2012xi}.

\subsection{Searches for particles with telescopes}
There are several ways that particles with a fractional charge could be produced in cosmological events;

\vspace{2mm}
\begin{itemize}
\item the particles were produced early on in the Universe and are a stable component of the present matter;
\item the particles are rare but can be continuously produced in high-energy astrophysical events; or
\item the particles are produced in cosmic ray processes near Earth.
\end{itemize}
\vspace{2mm}

\noindent Because there is no clear preference for one of these possibilities, most telescope experiments express their search sensitivity as an incoming flux close to the detector in units of [cm$^{-2}$ s$^{-1}$ sr$^{-1}$]. This analysis has adopted the same search strategy and aims to improve on previous results. The most stringent upper limit was realized by the MACRO experiment found on the arXiv (not published, Ref. \cite{Ambrosio:2004ub}) that compares results from older searches and can be seen in Figure \ref{fig:upperlimits}. The best published result is set by Kamiokande II with upper limits of $2.1 \times 10^{-15}$ cm$^{-2}$ s$^{-1}$ sr$^{-1}$ and $2.3 \times 10^{-15}$ cm$^{-2}$ s$^{-1}$ sr$^{-1}$ for particles with charges $\frac{1}{3}$ and $\frac{2}{3}$ respectively \cite{Mori:1990kw}.

\begin{figure}
\centering
\includegraphics[width=0.7\textwidth]{chapter2/img/upperlimits_changed.png}
\caption{Upper limits on fluxes of particles close to the respective detectors. LIP stands for \textit{Lightly Ionizing Particles}. ``This search'' refers to an unpublished result from the MACRO experiment that compares the published results from other experiments. Figure taken from Ref. \cite{Ambrosio:2004ub}, where I have changed the Kamiokande results which are wrong in the original figure.}
\label{fig:upperlimits}
\end{figure}

%https://arxiv.org/pdf/1601.04004.pdf +references!

\section{Properties of the signal}
\label{sec:properties}
Because there are many possible scenarios what these particles are, where they originate from, or how they are produced, one has to make certain assumptions about the properties of the signal. A particle traveling at the speed of light with a lifetime < 0.1 seconds traversing a detector will not give the same signal properties as one that has a very long lifetime. Therefore, I have assumed that the particles I am looking for
\vspace{2mm}

\begin{itemize}
\item behave leptonically, similar to muons. The particles will therefore produce long tracks instead of cascades in the IceCube detector (more info in Chapter \ref{ch:cherenkov});
\item have a long lifetime and will not rapidly decay within the IceCube detector, or at least have a very low probability to do so;
\item follow an energy spectrum with a spectral index of -2\footnote{More information about spectra can be found in Section \ref{subsec:whatarecosmicrays}.};
\item are assumed to produce an isotropic flux close to the detector.
\end{itemize}
\vspace{2mm}

\noindent These assumptions are consistent with previous searches that are mentioned in Section \ref{sec:prevsearches}. The behavior of such particles in the detector will depend on the charge (see Section \ref{subsub:energyloss}) and, to a lesser extent, the mass. In this work the particles are assumed to have a
\vspace{2mm}

\begin{itemize}
\item charge of 1/3, 1/2 or 2/3;
\item mass of 10 GeV, 100 GeV, 1 TeV, 10 TeV or 100 TeV,
\end{itemize}
\vspace{2mm}

\noindent where I have referred to the charge of the particles as relative to the absolute electron charge, $e$\footnote{This will be done throughout this work from this point on.}. The possible combinations result into a total of 15 unique signal samples that will be searched for.\\

\noindent The detector used in this analysis, the IceCube detector, is a neutrino telescope. The properties of the signal were defined to be able to compare this work to the results seen in Figure \ref{fig:upperlimits}. The best limits were set by the MACRO and Kamiokande experiments. The former was a multipurpose underground detector located at Gran Sasso, Italy and designed to search for magnetic monopoles with the use of liquid scintillators and streamer tubes \cite{Giacomelli:2007sk}. This design also made it possible to operate as a neutrino detector and cosmic ray observatory and allowed for BSM searches such as particles with a fractional charge.

The Kamiokande observatory, a water Cherenkov detector (see Chapter \ref{ch:cherenkov}) was designed to search for proton decay and located deep underground to shield the detector from cosmic ray muons (see Chapter \ref{ch:cr}). Because of its design and very efficient shielding from cosmic rays, the detector also operates as a neutrino observatory and can also be used to search for particles with a fractional charge. The experiment was later extended to Kamiokande II, III and Super-Kamiokande.  