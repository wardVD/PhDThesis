\begin{appendices}
\chapter{Gauge symmetries}
The difference between global and local symmetries are not straightforward for everybody. In this appendix I try to give a better view of the matter.

Imagine that at each point in space and time there is a circle attached to it. If one shifts all circles of all points with a fixed angle the underlying physics hasn't changed. If we look at the whole in a different angle, nothing seems to be changed as everything holds the same relative orientation. This is a global symmetry. For local symmeries we instead shift each circle through a different angle, but an angle that changes smoothly from point to point and in a way that we can say how that angle is varying between different nearby regions. Then it will turn out that we can describe that rotation angle by means of a so-called gauge field, which just lets us transport the charged scalar field from one point in space time to another, taking account of how the rotation angle of the circle is changing. A gauge is a kind of coordinate system that is varying depending on the location with respect to some underlying space. In physics we are almost always concerned with space-time as the underlying space, and we are typically interested in theories that are invariant with respect to the choice of gauge or coordinate system. 

Dan wat uitleg vanuit je QFT boek en de dingen hieronder:
Je wilt je derivative anders doen werken in je theory onder een transformatie, maar daarvoor heb je een veld nodig. M.a.w.: dankzij een veld heb je lokale ijktransformatie mogelijk!


%Now consider if one wished to not only make global changes of phase but also local gauge transformations. Then the phase α would have to become a spatial function, α(x). Note now that the kinetic term would pick up derivatives of α(x), so the action is not automatically invariant under this change. In order to make it invariant and enforce such a symmetry, one must rewrite the transformation law so that there is a new type of derivative Dμϕ, which under the change of phase on ϕ transforms in the same fashion,
%Dμϕ→eiα(x)Dμϕ.
%This derivative is said to be a gauge covariant derivative, in that it varies in the fashion that the field does when a gauge transformation is performed. To construct such a derivative, one introduces a new field in space called the gauge field whose transformation law under the gauge transformation will be prescribed so as to give the above law. The gauge derivative then becomes
%Dμϕ=∂μϕ+igAμϕ.
%(Note that the gauge derivative acting on ϕ∗ has the opposite sign in front of i, from the complex conjugate of this equation). The gauge field is playing the role of a connection field. In order for the derivative to be gauge covariant, the gauge field must transform as
%Aμ→Aμ+1g∂μα. 


\chapter{Planck's law}
\label{ch:planck}
bron: http://hyperphysics.phy-astr.gsu.edu/hbase/quantum/rayj.html
\section{Electromagnetic waves in a cubical cavity}
Suppose we have EM waves in a cavity at equilibrium with its surroundings. These waves must satisfy the wave equation in three dimensions:

\begin{equation}
\label{eq:wave}
\frac{\partial^2 \Psi}{\partial x^2} + \frac{\partial^2 \Psi}{\partial y^2} + \frac{\partial^2 \Psi}{\partial z^2} = \frac{1}{c^2} \frac{\partial^2 \Psi}{\partial t^2}.
\end{equation}
The solution must give zero amplitude at the walls. A non-zero value would mean energy is dissipated through the walls which is in contradiction to our equilibrium assumption. A general solution takes the form of

\begin{equation}
\Psi(x,y,z,t) = \Psi_0 \sin{k_1x} \sin{k_2y} \sin{k_3z} \sin{k_4 t},
\end{equation}
which, after requiring $k_n L = n \pi$ with $n=0,1,2...$ and $k_4 \frac{\lambda}{2c} = \pi$, leads to

\begin{equation}
\Psi(x,y,z,t) = \Psi_0 \sin{\left(\frac{n_1 \pi x}{L}\right)} \sin{\left(\frac{n_2 \pi y}{L}\right)} \sin{\left(\frac{n_3 \pi z}{L}\right)} \sin{\left(\frac{2\pi ct}{\lambda}\right)}.
\end{equation}

From the wave equation it is easy to find that

\begin{equation}
\label{eq:n}
n^2 = n_1^2 + n_2^2 + n_3^2 = \frac{4L^2}{\lambda^2},
\end{equation}
which span up a sphere in ``n-space'' with a volume of $\frac{1}{8}\frac{4}{3}\pi n^{3/2}$, where the first term originates from the positive nature of $n_{1,2,3}$. Because there are two possible polarizations of the waves one has to multiply with an additional factor 2. The number of modes per unit wavelength is equal to

\begin{equation}
\frac{dN}{d\lambda} \times \frac{1}{L^3} = \frac{d}{d\lambda}\left[\frac{8\pi L^3}{3\lambda^3}\right]  \times \frac{1}{L^3} = - \left[\frac{8\pi}{\lambda^4}\right].
\end{equation}
\subsection{Classical approach}
Following the principle of equipartition of energy, each standing wave mode will have an average enkergy $kT$ with $k$ the Boltzmann constant and $T$ the temperature in Kelvin. The energy density is then:

\begin{equation}
\frac{du}{d\lambda} = -kT \frac{8\pi}{\lambda^4}.
\end{equation}
In function of frequency $\nu = \frac{c}{\lambda}$:

\begin{equation}
\label{eq:rayleigh}
\frac{du}{d\nu} = -\frac{c}{\lambda^2} \frac{du}{d\lambda} = \frac{8\pi kT \nu^2}{c^3},
\end{equation}
also known as the Rayleigh-Jeans law\footnote{This is often quoted per unit of staradian, which results in $\frac{2 kT \nu^2}{c^3}$}. Problem: divergence

\subsection{Quantum approach}
The energy levels from a quantized harmonic oscillator are equal to

\begin{equation}
E_r = h\nu\left(r+\frac{1}{2}\right) = \frac{hc}{\lambda}\left(r+\frac{1}{2}\right) \ \ \ \textrm{with} r = 0,1,2,...
\end{equation}
Implementing eq. \ref{eq:n}

\begin{equation}
E = \left(r+\frac{1}{2}\right) \frac{hc}{2L}\sqrt{n_1^2 + n_2^2 + n_3^2}
\end{equation}
According to statistical physics the average energy is now not equal to $kT$ but follows a probability distribution

\begin{equation}
p(\nu,r) = \frac{e^{\frac{-rh\nu}{kT}}}{\sum_{r=0}^\infty e^{\frac{-rh\nu}{kT}}},
\end{equation}
where we reference to the ground state of the oscillator: $E'_r = E_r - E_0$.

The average energy is now:

\begin{equation}
\begin{split}
\langle E(\nu)\rangle &= \sum_{r=0}^\infty E(\nu,r) \cdot p(\nu,r) = \frac{\sum_{r=0}^\infty r h \nu \  e^{\frac{-rh\nu}{kT}}}{\sum_{r=0}^\infty e^{\frac{-rh\nu}{kT}}}\\
&= \frac{h\nu}{e^{h\nu/kT} - 1}
\end{split}
\end{equation}
Substituting this for $kT$ in eq. \ref{eq:rayleigh} we find Planck's equation:

\begin{equation}
\frac{du}{d\nu} = \frac{8\pi h \nu^3}{c^3} \frac{h\nu}{e^{h\nu/kT} - 1}
\end{equation}

\end{appendices}