\chapter*{Introduction}
\label{introduction}
\addcontentsline{toc}{chapter}{\protect\numberline{}Introduction}%

\textit{\begin{flushright} Scientists have become the bearers of the torch of discovery in our quest for knowledge $\sim$ Stephen Hawking
\end{flushright}}

\noindent Having a quest for knowledge seems to be a trait that belongs to human kind. The last couple of centuries, there was a transition from providing simple explanations of physical phenomena to \textit{really understanding} the problem and properly describing it. More and more, a critical mindset became important in having new insights; being convinced of having the final answer is no longer enough. Theories are tested and are only accepted if they can be reproduced in experiments. This approach is not the most easy one and more than often, scientists encounter many obsticles along their way. However, I believe that the feeling scientists get when they have solved a puzzle and see their theory confirmed in an experiment is the driving force behind new discoveries and pushing the limits of current theories. Wanting to know what is beyond our current understanding has made us discover our planet (and others) and never stops us to come up with new questions.\\

\noindent Experimental particle physics is a field that tries to find the smallest detectable particles that make up matter and radiation. These particles carry certain properties such as mass, charge, spin, etc. and allow them to interact with each other. There are only a handful of fundamental interactions know to date: gravity, the strong force, the weak force and the electromagnetic force. It appears that nature follows certain physical rules that can be expressed in a mathematical form. Mathematics made it possible to write down our findings and to expand our knowledge. The theory that describes the last three of the four forces and classifies all known fundamental particles is called the Standard Model (SM) of particle physics. In Chapter \ref{ch:SM}, an overview is given of the properties of particles and interactions and ends with why fysicists think the SM in its current form cannot be the final answer.\\

\noindent There are many theories that try to expand the SM and are called ``Beyond-the-Standard-Model theories'' (BSM). They introduce new symmetries, new particles, new interactions, etc. Some of these theories also predict the existence of free particles with an electromagnetic charge that is lower than the electron charge, $e$. Quarks are known to have charges lower than $e$, but cannot exist freely. In Chapter \ref{ch:theoreticalmotivation} we motivate the possible existence of these particles and describe their assumed properties.\\

\noindent With the discovery of cosmic rays, a new field of physics was born: astroparticle physics. Experiments could be done on particles that could not originate from our solar system. The energy of these particles also has a wide range and offers to examine particles with a very high energy that do not have to be accelerated by man-made objects. For example, positrons, muons, pions, and kaons were first discovered in cosmic ray interactions. Cosmic rays play a crucial role in modern multimessenger astronomy along with gamma rays, gravitational waves and neutrinos. Each one of these fields has its strengths and weaknesses and many different experiments try to look for these respective signatures. Luckily, some of these experiments can also be used to look for BSM physics but have to account for the processes these experiments are designed to be sensitive too. Therefore, in Chapter \ref{ch:cr}, we give an overview of the physical processes that are visible in neutrino detectors. We will describe the origin of cosmic rays and neutrinos and how showers of particles are produced from cosmic ray interactions.\\

\noindent Charged particles can be detected by the production of Cherenkov radiation. These particles can be exotic particles, particles produced in cosmic air showers or secondaries that are created from neutrino interactions. The Cherenkov effect is explained in Chapter \ref{ch:cherenkov}, together with how charged particles lose their energy when traveling through matter.
To be able to detect high energetic neutrinos, it is crucial to make the instrumented volume of a detector as large as possible. In Chapter \ref{ch:icecube}, we describe how the IceCube Neutrino Observatory is constructed in the ice at South Pole. We describe the hardware components and how the ice is used and modeled for neutrino and charged particle detection.

The next part of this work has a larger focus on the technical details of this analysis that is dedicated to search for particles with an anomalous charge. The simulation steps are discussed in Chapter \ref{ch:simulation} and several analysis techniques that were used in this work are explained in Chapter \ref{ch:reconstruction}. In Chapter \ref{ch:space} an overview of the analysis is given and how it was possible to discriminate background events from events that originate from particles with an anomalous charge. We finalize with our findings if these particles exist or if no new physics is found, their maximal abundance that is measurable in the IceCube detector is shown.

Chapter \ref{ch:summary} provides a summary and discussion of the work.