\chapter*{Nederlandstalige Samenvatting}\label{nederlands}
\addcontentsline{toc}{chapter}{\protect\numberline{}Nederlandstalige samenvatting}%

\vspace{-5mm}
\textit{\begin{flushright}
Denkt aleer gij doende zijt en doende, denk dan nog $\sim$ Guido Gezelle
\end{flushright}}

\noindent De aanpak om nieuwe dingen te ontdekken binnen de fysica is de laatste eeuwen sterk ge\"evolueerd. De overgang van eenvoudige verklaringen naar theorie\"en die fenomenen niet alleen konden uitleggen, maar ook experimenteel konden reproduceren, bleek een cruciale stap naar een nieuwe methode in fysisch onderzoek. Als gevolg bracht deze methode nieuwe voorspellingen met zich mee; een theorie kon iets verklaren maar ook fenomenen beschrijven die tot dan toe nog niet werden geobserveerd. Deze fenomenen werden nadien ofwel waargenomen ofwel werd de theorie weerlegd. Hieruit volgde een natuurlijke evolutie in wetenschappelijke aanpak waarbij theorie en experiment elkaar horen te bevestigen.

In de loop van de 20ste eeuw viel er door de technologische ontwikkelingen veel te ontdekken. Men begon langzamerhand beter te begrijpen hoe de wereld in elkaar zit en hoe de wetten van de fysica neergeschreven konden worden. Wiskunde was daarin de taal van fysici om een theorie te ontwikkelen en experimenten te beschrijven. Experimentele deeltjesfysica kende zijn (eerste) hoogdagen in de jaren '60 waarbij deeltjesversnellers het ene na het andere nieuwe deeltje ontdekten: de ``particle zoo''. Fysici drongen aan op een theorie die een beter beeld kon geven; een handvol basisstenen (fundamentele deeltjes) en regels hoe deze deeltjes zich gedragen en hoe ze tot uiting komen. Die theorie werd de daaropvolgende jaren ontwikkeld en is gekend als het \textit{standaardmodel van de deeltjesfysica}. Het blijkt tot op heden een van de meest succesvolle ooit binnen het vakdomein van de fysica.\\

\noindent Met de ontdekking van het Brout-Englert-Higgs deeltje in 2012 werd de laatste bouwsteen, die door het standaardmodel voorspeld was, gevonden. Het model verklaart echter niet alles. We hebben bijvoorbeeld nog geen idee wat donkere materie precies is, waarom er zoveel minder antimaterie is in vergelijking met materie, enzovoort. Ook de schijnbaar wiskundige willekeur in het standaardmodel roept tot op heden nog steeds vragen op. Er lijken drie types leptonen te zijn. Waarom drie? En waarom zijn er welbepaalde symmetrie\"en die het standaardmodel beschrijven terwijl er ontelbaar veel andere symmetrie\"en mogelijk zijn? Waarom deze niet? Om deze redenen proberen fysici in experimenten op zoek te gaan naar processen die met de huidige, aanvaarde modellen nog niet beschreven kunnen worden. Dit zou ons een betere controle kunnen geven welk soort nieuwe theorie - waar er honderden van zijn - de juiste is. Deze experimenten worden steeds complexer: ze worden groter of minutieuzer en veelal een combinatie van beide. Deze complexiteit brengt een groter kostenplaatje met zich mee waardoor deze projecten goed beargumenteerbaar moeten zijn: ze zijn een gerichte tast in het duister.

Er zijn ook andere experimenten die met de waarnemingen van gekende deeltjes proberen om beter te achterhalen hoe bepaalde processen in ons universum tot stand komen. Een voorbeeld hiervan is het IceCube neutrino observatorium dat gesitueerd is in het ijs centraal op de Zuidpool. Het voornaamste doel van dit experiment is om neutrino's te detecteren, deeltjes die zelden met materie interageren en net daarom een grote bron aan informatie bevatten omdat ze veel vertellen over hun oorsprong. We weten tot op heden nog maar weinig over de precieze werking van de bronnen van deze neutrino's.

Om zoveel mogelijk gebruik te maken van bestaande infrastructuur worden bestaande detectoren zoals IceCube ook gebruikt in zoektochten naar nieuwe fysica. Het doel van dit werk is om met behulp van het IceCube experiment na te gaan of er deeltjes bestaan met een elektromagnetische lading die het standaardmodel niet voorspelt.\\

\noindent Het IceCube experiment werkt met behulp van lichtgevoelige modules die per 60 aan een kabel bevestigd zijn. 86 van zo'n kabels zijn verspreid in een hexagonaal vlak in het ijs dat in totaal ongeveer een kubieke kilometer van volume inneemt. Geladen deeltjes zoals elektronen en muonen die uit een neutrino-interactie kunnen komen, produceren licht in het ijs dat door deze modules waargenomen kan worden. Dit fenomeen is gekend als het Cherenkov-effect en zegt dat de hoeveelheid licht dat geproduceerd wordt afhangt van de lading van het deeltje. Aangezien de enige vrije deeltjes die we kennen een lading hebben die een veelvoud is van de elementaire lading van een elektron $e$, is het in theorie mogelijk om een onderscheid te maken tussen de gekende deeltjes en deeltjes met een lading die lager is dan $e$. Deze laatste maken geen deel uit van het standaardmodel en hun observatie zou een nieuwe start kunnen geven voor een meer omvattende theorie die mogelijks andere onopgeloste vragen helpt te verklaren.\\

\noindent De voornaamste uitdaging om op zoek te gaan naar dergelijke deeltjes blijkt de beperking van de detector zelf te zijn. Gekende deeltjes met een hele lage energie blijken niet eenvoudig te onderscheiden van deeltjes met een lage lading. Dit komt voornamelijk omdat de modules van de detector minimaal 17 meter van elkaar liggen. Hierdoor gaat veel van het licht verloren omdat het de optische instrumenten niet kan bereiken. In dit werk wordt er beschreven hoe deze nieuwe deeltjes onderscheiden kunnen worden van muonen en elektronen. Deze laatste ontstaan uit botsingen van kosmische straling met onze atmosfeer waar ze veelvuldig worden geproduceerd in zogenoemde ``air showers''. Ze kunnen ook de detector betreden als secundaire deeltjes nadat neutrino's interageren met het ijs. Deze neutrino's kunnen geproduceerd worden in kosmische processen zoals gamma ray bursts of supernova's, maar zijn er vooral in grote getale aanwezig door de voorgenoemde air showers.\\

\noindent De analyse beschrijft hoe er vanuit de IceCube data een selectie gemaakt kan worden om deze nieuwe fysica - als ze er is - zo goed mogelijk zichtbaar te maken. Na een reeks selecties om de kwaliteit van de data te verhogen, werden meerdere variabelen gebruikt en ontwikkeld om nadien ge\"implementeerd te worden in een ``Boosted Decision Tree'', i.e. een machine learning techniek die wordt gebruikt in datamining. Er werd tevens ook gebruik gemaakt van een resampling techniek genaamd ``pull-validation'' om de beperkte statistische mogelijkheden beter te handhaven. De enige mogelijke conclusie die getrokken kon worden was dat er geen indicatie was voor de aanwezigheid van deeltjes met een lage lading. Er werd een bovenlimiet opgesteld waarbij ook rekening gehouden werd met meerdere onzekerheden die deze limiet zouden kunnen be\"invloeden zoals de eigenschappen van het ijs waardoor het licht schijnt. Deze limieten zijn een verbetering van voorgaande experimenten en werden berekend volgens de techniek van Feldman en Cousins binnen een 90\% betrouwbaarheidsinterval.\\

\noindent Dit werk was een eerste poging om gebruik te maken van het IceCube experiment om op zoek te gaan naar nieuwe fysica in de vorm van deeltjes met een lading die tot op heden nog nooit zijn geobserveerd. Dergelijke deeltjes werden niet waargenomen waardoor een bovenlimiet werd opgesteld in de mate van hun aanwezigheid mochten ze weldegelijk bestaan.